\documentclass{ctexart}
\usepackage{hyperref}
\usepackage[margin=2cm]{geometry}

\begin{document}

\pagestyle{empty}

\section*{A - 你好,所以,再见~}

数据结构入门题,做法有很多,包括但不限于线段树、分块(可能会被卡)、树状数组(比较 Tricky 但是很值得学习)、CDQ 分治、莫队(也可能会被卡)等。在此不再细述。如果还没有掌握这些知识点的同学可以去翻阅网络上的资料进行学习。

大概看了一眼没过的代码,没通过的大概有一下原因:

\begin{itemize}
    \item (WA)没有开 \texttt{ll}:这道题每次询问的答案最大值为 \(10^6\times10^6\times10^6=10^{18}\),开 \texttt{int} 是肯定存不下的。除此以外,许多计算过程的中间量也需要开 \texttt{int}。(Tips:使用 \texttt{\#define int long long} 可能是个不错的选择,但还是建议学会分析什么时候需要开 \texttt{ll})
    \item (TLE)没写懒标记/懒标记不起效:有些同学写线段树没写懒标记,或者懒标记的写法有误,导致单次修改的复杂度退化为 \(O(n)\)。
    \item (TLE)使用朴素算法:复杂度分析是非常重要的一个知识点。如果你还不会分析一个算法的复杂度及其运行时间估算,可以参考 \href{https://oi-wiki.org/basic/complexity/}{OI-wiki} 进行学习。
    \item (RE)数组开小了:不要抠抠搜搜的,请把数组开大些。(线段树要开四倍空间,请牢记)
    \item (WA)细节写错了:如果你很长时间都没找到你哪里写错了,可以试着和一个朴素算法的代码进行\textbf{对拍}。至于如何对拍可参考 \href{https://oi-wiki.org/contest/common-tricks/#\%E5\%AF\%B9\%E6\%8B\%8D}{OI-wiki} 上的相关章节,在此不再细述。
\end{itemize}

最后,请不要读入数组 \(\lbrace a_i\rbrace\)。

\end{document}
